\documentclass[12pt, draft]{article}

\begin{document}
\title{A Population-Based Measure for District Compactness, Explored in North Carolina}
\author{Nicholas Miklaucic}
\date{May 2018}
\maketitle

\section{Introduction}
Gerrymandering in America has significant political effects, and efforts to mitigate partisan
gerrymandering have been hindered by the lack of consensus on effective quantification of the
problem.

This paper will focus on compactness as a measure of gerrymandering: using district shape rather
than election data to quantify the extent to which natural divisions of a state have been eschewed
in favor of politically-expedient shapes. This approach identifies with a common ``I know it when I
see it'' approach to gerrymandering in public discourse on the subject.

For example, the post-2010 Census NC congressional district plan, which was struck down for racial
gerrymandering in \textit{Cooper v. Harris}, drew incredulity for the shape of its 1st and 12th
districts. The 1st was described as ``akin to a Rorschach ink blot'' by one district resident.

This paper proposes and expounds on a relatively novel compactness measure based on population
distribution within districts that attempts to accurately measure the extent to which natural
communities of interest have been distorted for political or racial gain. The optimization of this
algorithm for a state is then explored, and through new techniques effective heuristic optimizations
of this measure result in proposed districting plans that are then compared with the existing and
prior ones using other compactness measures.

\section{The Districting Problem}
Define a discrete collection of $n$ voter groups $B = \{b_1, b_2, \cdots, b_n\}$, each with a location
and a population. (In reality, these are likely to represent census blocks or some other unit of
data, and it is assumed that $n$ is large enough that the approximation of assuming each of these
units is grouped entirely at the centroid is sufficiently accurate.)

We define a \textbf{districting plan} with $k$ districts as an assignment of each block $b_i$ to a
district $D_1, D_2, \cdots, D_k$. Several constraints on a feasible districting plan are already
well-established: each will be considered in turn.

\subsection{Contiguity}

\end{document}
